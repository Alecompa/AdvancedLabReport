The scope of TRACE array coupled to GALILEO, complemented by the array of 6 large volume scintillator detectors (LaBr$_{3}$(Ce)), to investigate the possibility of measuring, for the first time, the $\gamma$ decay from the 7-keV narrow resonance in $^{19}$O, located at 4109 keV, i.e., 153 keV above the neutron-decay threshold. Also, the study of the $\gamma$ decay from other narrow resonances in $^{19}$O (for which the $\gamma$ decay is expected to compete with neutron and alpha emission, with branchings of the order of 10$^{-3}$ – 10$^{-4}$) will be considered.

The main idea for reconstructing the excitation energy of $^{19}$O (i.e., to display the narrow resonances of interest) is to detect the evaporated protons, isotropically emitted in the center of mass reference system with an estimated total cross section of about 50 mb. For this purpose, a segmented light charged particle array was employed, made of 4$\Delta$E-E telescopes from the TRACE project, positioned at backward angles with respect to the beam direction, coupled to the GALILEO detection system.

This report presents the initial stages of the TRACE calibration with 3 peaks from 3 different $\alpha$ source ( ), the construction of the electronical chain for the data acquisition and the development of a Neural Network (NN) model, based on the Pulse Shape Anlaysis (PSA) of the signal from the detector, in order to identify the proton and $\alpha$ events. Finally, an initial and brief attempt for the $\gamma$ identification is presented.