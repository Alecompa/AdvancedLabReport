\documentclass[a4paper, 11pt]{article}

\usepackage[english]{babel}
\usepackage[utf8]{inputenc}
\usepackage[T1]{fontenc}
\usepackage{graphicx}
\usepackage{color}
\usepackage{amsmath,amssymb}
\usepackage{rotating} 
\usepackage{layaureo}
\usepackage{booktabs}
\usepackage{varioref}
%\usepackage{subfigure}
\usepackage{listings}
\usepackage{wrapfig}
\usepackage{siunitx}
\usepackage{physics}
\usepackage{subcaption} 
\usepackage{subfloat}
\usepackage{caption}
\usepackage{gensymb}

\sisetup{output-decimal-marker={.}}

\author{Jakub Skowronski & Alessandro Compagnucci}
\title{PROTO Trace}

\begin{document}


\maketitle

\section{Introduction}

\section{Physics}

\section{Data Acquisition}

In order to prepare the data acquistion system, the entire electronical chain
was tested beforehand. In fact it is crucial that the response of each
electronic device used behaves linearly, as the information about the energy
of the incident particle must be univocally extracted at the end of the chain.
To test the PROTO Trace detector, it was put in a vacuum chamber, with a
$\alpha$-ray source

\subsection{PROTO Trace detector}

\subsection{Electronical Chain}
         
\end{document}
