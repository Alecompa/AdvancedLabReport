An in-depth characterization and calibration of the TRACE detector was given using an alpha source. The detector was then tested in the GALTRACE experiment at LNL. The data acquired during the different runs was then analyzed in order to reconstruct the structure of each event comprehensive of the PSA variables extracted from the raw signals from the digitizer. Fig.~\ref{imax} and Fig.~\ref{risetime} shows the plot obtained from this analysis.


Then, was also showed that is possible to train a neural network using signals from different regions of the graph to distinguish between different types of particles (Section~\ref{deep}).


A full reconstruction of the events as well as a selection of interesting transition is behind the scope of this work, however some work was made in order to characterize the spectrum obtained by the GALILEO array coupled with the TRACE detector (Section~\ref{gammaspectr}).
