One of the modern detection methods, offering identification of the reaction
products with very low energy, is the Pulse Shape Analysis (PSA). This method
can be applied to the signals from silicon detectors. As shown by Mengoni et
al.~\cite{mengoni}, in the case of the TRacking Array for Charged Ejectile (TRACE) array, consisting
of 200-$\mu$m thick silicon modules and divided in 60 separately read pixels,
the identification of the \ce{^{1,2,3} H} isotopes can be easily obtained.
In addition, the separation between \ce{^3 He} and \ce{^4 He} was also
observed. This proved that the thin detector, with the uniformity guaranteed
by the fine pad segmentation, may provide a good particle discrimination when
the PSA technique is applied.

\bigbreak

The experimental application choosen to test the detector is the reconstruction of the excitation energy of \ce{^19 O},
a neutron-rich nuclei, by the detection of the evaporated protons,
isotropically emitted in the center of mass reference system with an estimated total cross section of about $50$ mb. For this purpose, a segmented light
charged particle array was employed, made of 4$\Delta$E-E telescopes from the
TRACE project, positioned at backward angles with respect to
the beam direction, coupled to the GALILEO detection system~\cite{galileo}.
If successful, this would allow to investigate the structure of light reaction
products, such as \ce{Be}, \ce{B}, \ce{C}, \ce{N} and \ce{O} by a direct
measurement of their energy, position, mass, and charge.

\bigbreak

This report presents the initial stages of the TRACE detector calibration
with 3 peaks from 3 different $\alpha$ source (\ce{^239 Pu}, \ce{^241 Am},
\ce{^244 Cm}), the construction of the electronical chain for the data
acquisition and the development of a Neural Network (NN) model, based on the
Pulse Shape Anlaysis (PSA) of the signal from the detector, in order to
identify the proton and $\alpha$ particles emmited by the daughter nuclei
produced during the experiment. Finally, an initial and brief attempt for the
$\gamma-\gamma$ coincidences identification is presented.